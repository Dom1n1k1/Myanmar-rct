% Myanmar Mobile Micro Health Insurance RCT — Paper Skeleton
% Last updated: 2025-08-14

\documentclass[12pt]{article}

% ---------- Packages ----------
% Encoding, fonts, typography (use these with pdfLaTeX)
\usepackage[T1]{fontenc}
\usepackage[utf8]{inputenc}
\usepackage{lmodern}
\usepackage{microtype}

% Line numbers (enable with \linenumbers after \begin{document})
\usepackage[left]{lineno}
\modulolinenumbers[1] % continuous numbering

% Math + patching for lineno in math environments
\usepackage{amsmath,amssymb}
\usepackage{etoolbox}
\AtBeginEnvironment{align}{\linenomath}
\AtEndEnvironment{align}{\endlinenomath}
\AtBeginEnvironment{align*}{\linenomath}
\AtEndEnvironment{align*}{\endlinenomath}
\AtBeginEnvironment{equation}{\linenomath}
\AtEndEnvironment{equation}{\endlinenomath}
\AtBeginEnvironment{gather}{\linenomath}
\AtEndEnvironment{gather}{\endlinenomath}

% Tables and figures
\usepackage{booktabs}
\usepackage{dcolumn}
\usepackage[labelfont=bf,font=small]{caption}
\usepackage{makecell}
\usepackage{array}
\usepackage{float}
\usepackage{graphicx}
\usepackage{rotating}

% Numbers/units
\usepackage{siunitx}
\sisetup{
  group-digits=false,
  input-symbols = ( ) * + -,
  table-align-text-post = false,
  table-number-alignment = center,
  detect-weight = true,
  detect-family = true,
  table-space-text-post={***}
}

% Page layout and spacing
\usepackage{geometry}
\geometry{margin=1in}
\usepackage{setspace}
\onehalfspacing

% Graphics / diagrams
\usepackage{tikz}
\usetikzlibrary{shapes.geometric,arrows.meta,positioning}
\tikzstyle{box} = [rectangle,draw,rounded corners,minimum width=3.5cm,minimum height=1.1cm,text centered,fill=white]
\tikzstyle{arrow} = [thick,->,>=stealth]

% Misc
\usepackage{ragged2e}
\usepackage{xurl}
\urlstyle{same}

% Bibliography (biblatex-apa) — load BEFORE hyperref
\usepackage[
  backend=biber,
  style=apa,
  citestyle=apa,
  natbib=true,
  sorting=nyt,
  maxcitenames=2,
  maxbibnames=99,
  doi=true,
  url=false,   % <— turn off URLs for APA journal articles
  urldate=long,
  eprint=false
]{biblatex}
\DeclareLanguageMapping{english}{english-apa}
\addbibresource{refs.bib}

% Hyperlinks — load AFTER biblatex
\usepackage{hyperref}
\hypersetup{
  colorlinks=true,
  linkcolor=blue,
  citecolor=blue,
  urlcolor=blue
}

% ---------- Title ----------
\title{Mobile Health Microinsurance and Health Care Access in Myanmar: \\
Evidence from a Randomized Controlled Trial}
\author{Author Name\thanks{Affiliation, email. Acknowledgments here as needed.}}
\date{\today}

% ---------- Document ----------
\begin{document}
\maketitle
\thispagestyle{empty}

\begin{abstract}
% 150–200 words. Brief context, intervention, design, main results, and policy takeaway.
% If results are null due to implementation shortfalls, say so clearly and explain how you document exposure, take-up, and timing.
\end{abstract}

\noindent\textbf{Keywords:} micro health insurance; telemedicine; out-of-pocket expenditures; randomized controlled trial; Myanmar

\noindent\textbf{JEL codes:} I13; I15; O12; C93

\newpage
\setcounter{page}{1}
\onehalfspacing

% ========== 1. Introduction ==========
\section{Introduction}
% Context: UHC challenges in Myanmar and the role of micro health insurance.
% Brief description of Bright Start and the MHMI package.
% Contribution: experimental evidence in a fragile setting with careful documentation of implementation and exposure.
% Roadmap of the paper.

% ========== 2. Intervention and Theory of Change ==========
\section{Intervention and Theory of Change}
% Describe the MHMI package: hospital insurance, telemedicine, outpatient benefit, information messages.
% Present the theory of change from the protocol annex.
% Map inputs -> activities -> outputs -> outcomes (utilization, financial protection).

% ========== 3. Study Design ==========
\section{Study Design}
\subsection{Setting and Sample}
% Two peri-urban townships in Yangon: Hlaing Thar Yar and Shwe Pyi Thar.
% Target population: pregnant women and children under six, recruited from partner lists.

\subsection{Randomization and Assignment}
% Unit of randomization: household. Allocation 1:1 to treatment and control.
% Geographic clustering considerations.

\subsection{Data Collection}
% Baseline phone survey, planned midline, endline survey about one year later.
% Survey instruments and timing. Consent and ethical oversight.

\subsection{Outcomes}
% Primary: telehealth use, outpatient use, OOP expenditures.
% Secondary: hospitalizations, information access, financial resilience.

\subsection{Power Calculations}
% Summarize assumptions: baseline utilization around 12–15 percent, ICC around 0.3, MDEs around 5 percentage points for utilization, 1.5 percentage points for hospitalizations, 5–6 USD drop in OOP.
% Discuss implications for detectable effects under attrition.

% ========== 4. Implementation and Exposure ==========
\section{Implementation and Exposure}
% Planned rollout versus realized rollout.
% Enrollment flow, timing alignment between treatment activation and survey waves.
% Take-up and adherence metrics. Barriers and context shocks.
% Administrative data status and attempts to obtain it.

% ========== 5. Empirical Strategy ==========
\section{Empirical Strategy}
\subsection{Intent-to-Treat Estimation}
% Specify ITT model: outcome on treatment assignment with pre-specified controls, township fixed effects if planned, and clustering at the household level.
% Outline outcomes and time windows.

\subsection{Exposure and Treatment-on-the-Treated (Exploratory)}
% If available, define exposure measures from survey self-reports.
% Caution about selection. Present as exploratory.

\subsection{Multiple Outcomes and Inference}
% Pre-specify primary outcomes. Discuss adjustment strategy for families of outcomes if needed.

\subsection{Heterogeneity Analyses}
% Township, baseline vulnerability, baseline utilization, child versus pregnant woman.

\subsection{Missing Data and Attrition}
% Balance checks at baseline. Differential attrition tests. Weighting or bounding if needed.

% ========== 6. Results ==========
\section{Results}
\subsection{Descriptive Statistics and Balance}
% Table of baseline characteristics by assignment. Standardized differences.

\subsection{Implementation and Take-up}
% Graphs or tables documenting exposure timelines and take-up rates.

\subsection{Main ITT Effects}
% Telemedicine and outpatient utilization.
% OOP expenditures.
% Hospitalization outcomes.

\subsection{Heterogeneity}
% Report a small set of pre-specified interactions.

\subsection{Robustness and Sensitivity}
% Alternative specifications, trimming, winsorizing OOP, clustering alternatives, randomization inference if applicable.

% ========== 7. Discussion ==========
\section{Discussion}
% Interpret findings given implementation realities.
% External validity and policy implications for micro insurance and telemedicine in fragile contexts.
% Lessons on timing alignment, data integration, and partner coordination.

% ========== 8. Conclusion ==========
\section{Conclusion}
% Concise summary, policy relevance, and research agenda.

% ========== Acknowledgments, Funding, Ethics ==========
\section*{Acknowledgments}
% Thank partners and field teams. List donors. Disclaimer text if required.

\section*{Funding}
% Identify UNICEF, Terre des hommes, and others as appropriate.

\section*{Ethics and Pre-Analysis}
% IRB approvals, protocol IDs, and any pre-analysis or registrations.
% Data security and consent procedures.

\section*{Data and Code Availability}
% Plan for sharing de-identified data, code, and replication package.
% Provide repository link once available.

% ========== References ==========
\printbibliography[heading=bibintoc,title={References}]

% ========== Appendix ==========
\appendix
\clearpage
\section{Additional Context and Protocol Details}
% Full protocol excerpts as needed. Figures of the theory of change.

\section{Survey Instruments and Variable Mapping}
% Map protocol outcomes to baseline, midline, and endline survey items.
% Include a table that links each construct to variable names.

\section{Power Calculations and Assumptions}
% Reproduce key power figures and assumptions. Document ICC choices and MDEs.

\section{Additional Tables and Figures}
% Balance tables by township. Robustness checks. Alternative outcome definitions.

\section{Administrative Data Request Log}
% Record of requests and responses from implementing partners.

\end{document}
